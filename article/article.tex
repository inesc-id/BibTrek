\documentclass{article}
\begin{document}
\title{\textbf{BibTrek Prototype: A Graph Visualization Tool and Search Engine for Cybersecurity Papers}}
\date{}
\maketitle

\section{Introduction}
(Quase final. Cortar palavras. Algumas das coisas aqui podem ser usadas como introduções na solução e avaliação)
\subsection{Problem}
As a result of the rising number of cyberattacks, it is becoming hard to find white-hat hackers that know how to thwart these kinds of attacks effectively. However, with the increased availability of information on the web it is becoming easier to study and learn how these attacks work. The cutting edge techniques used to carry them out is mostly found on the latest scientific research papers. Due to the many iterations of scientific knowledge the most recent work builds upon the work that has been done on the past. Because of this, to understand the most recent concepts it is important the reading of the referenced papers that made the writing of the current one possible. With this being said the most relevant paper of a certain subject becomes the paper with more references to it, thus making it an obligatory read. Having said that, the reality is that performing these kinds of bibliography surveys is time-consuming and inefficient.

\subsection{Solution}
The visualization of information using graphs is a benefit for our learning process and the construction of mental conceptual maps that makes it easier for us to retain information and relate it with everything we have learned so far on a certain topic.
With BibTrek we propose a graph visualization tool and search engine with the goal of relating the many different research papers in the area of cybersecurity and presenting that in an informative easy-to-browse graph using Neo4J. The use of this graph is therefore useful to relate the papers and their references making the workflow of the survey process much more efficient but also enjoyable. BibTrek would permit us to start a search from a set of either authors, publication titles or keywords dynamically producing a reference graph of with the most relevant scientific papers therefore permitting us an easy navigation and understanding of the relationships between every of its nodes.

\subsection{Overview}
In BibTrek to obtain the required information to populate our graph database we make use of the APIs provided to us by the major computer science libraries that host scientific research publications. After researching what the appropriate libraries were given their terms of service and API functionalities we have chosen as our means of obtaining information the libraries of: ArXiV, DBLP and IEEEXplore. Our app is presented through a simple command line Java program that through the use of user provided queries, crawls the libraries mentioned above using their APIs and adds to the Neo4J graph database all the available hosted information about either publications, institutions, authors and subjects. This graph database can also later be queried through the use of console commands to infer useful information about the state of the system. A simple example execution would follow these steps. Let us say we want to learn about the Spectre attack. We would choose the option to query one of the mentioned above libraries, choose the option to search for a publication and then type the word "Spectre" on the console terminal. After choosing the Spectre publication (written by ), the Spectre paper its references, authors and its subject will be added to the database and related with the other papers that were previously added by the user. After the fact, the user can simply watch the graph visualization created by BibTrek, this way finding which papers are the most referenced and therefore understand which shall be the one that he shall tackle firstly.

\section{Related Work}
#Yet to be researched

\section{Solution}
(Sketch)
#Write a brief introduction. Maybe use the text that I used above but here instead
\subsection{Design}
The BibTrek application was written using the Java programming language mainly due to the fact that it is ubiqutuous on modern-day computing systems. The database module uses Neo4J and requires us to have either a local or remote instance running it. The connection of the JVM to the Neo4J is made using the BOLT protocol on the network port: 7687. The API information is retrieved and parsed in the offered JSON format option due to its lightweight format and overall flexibility. The information in the database server is addressed using Cypher queries. The application was built using the Maven tool and uses the Neo4J driver repository in order to establish a connection with the database. It also uses the JSON repository in order to properly parse the JSON information retrieved in the GET responeses coming from the server. Finally it uses the java.net package in order to establish a connection with the DBLP API, query it and retrieve the response information.

\subsection{Implementation}
At the present time BibTrek only uses the DBLP API to obtain information about the publications. In the current implementation after the user chooses one of the options available to query the database, the user inputs a publication title that is scanned and properly parsed into a DBLP API formated query to the database. The queries are done through the URL link. After the query is sucessfully processed by the DBLP database the BibTrek HTTP module retrieves the response, parses the JSON and displays the information in the user console. After choosing whichever papers are appropriate for its research the system uses its writer module writing the parsed JSON response to a properly named ".nosql" file. The information is properly parsed into a "Cypher" query in order to be stored later in the Neo4J database Every request to the database creates a ".nosql" file that is read every 5 seconds by a thread that communicates with the Neo4J server app and stores the recent written data there. These files are then moved to a "logs" folder in order not to be re-written into the database. An example of a query from our system to the database would be the following. The user chooses the option to query the system by publication title. Let us say he choses the Spectre paper. Because of this a query is properly formated "dblp.org/publ/api?q=Spectre&format=json". In this example we are querying the DBLP database to fetch us all the available Spectre publications "q=Spectre" in the JSON format "format=json". After obtaining the HTTP response and parsing its JSON format the information is displayed to the user. Here every paper with the Spectre title is shown. The user chooses one or more of the papers to add to the database. After choosing the papers the information is then parsed into a cypher query and added to a nosql file. The thread responsible with the Neo4J communication and data storage will query the nosql folder for new files. If a new file is found the most recent data is then stored in the Neo4J communication. This information can then be visually seen by the user using the Neo4J browser. Finally the Neo4J Java thread moves the most recently written file to a logs folder in order for it not to be written again.

\section{Evaluation}
#Yet to be written

\end{document}
